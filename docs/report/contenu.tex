% Work in progress

\chapter*{Introduction}
	Dans le cadre du projet de Compilation, il nous a été demandé d'écrire un programme utilisant Ocamllex et Ocamlyacc destiné à générer une image au format SVG (utilisant XML). Le but de cet exercice est de nous confronter aux enjeux de l'analyse syntaxique et sémantique d'un langage source dans le but de le transformer en un langage cible.
	
	Ce rapport présente la méthodologie mise en place par notre groupe pour atteindre l'objectif du projet. Dans un premier temps, nous présenterons l'analyse du sujet que nous avons mené puis nous verrons le développement du compilateur à travers la grammaire reconnue. Pour finir, nous présenterons brièvement les langages source et cible manipulés par notre programme.

\chapter{Analyse du sujet}

	\section{Premières idées}
	Syntaxe
	
	\section{Mise en place de l'architecture}
	Structure de données utilisée pour stocker le xml 
	Construction du xml
	
	%TODO : parler de la génération qui se faisait à l'envers
	
	
\chapter{Développement du compilateur}
	
	\section{Partie 1 : La grammaire de base}

		\subsection{Construction de l'image}

		\subsection{Reconnaissance des formes basiques}

		\subsection{Ajout des attributs optionnels}
		% couleurs des formes, taille du texte, police d'écriture, etc...
	
		\subsection{Commentaires de fin de ligne}
	
		\subsection{Reconnaissance des formes complexes}
	
	\section{Partie 2 : Les instructions programmables}

		\subsection{Définition de variables et affectation}
		
		\subsection{Les opérations}

		\subsection{Les boucles}

\chapter{Présentation du langage final}

\chapter*{Conclusion}


